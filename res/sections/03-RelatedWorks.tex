\section{Related works}

Games are an instrument for entertainment and socialization. As described 
in~\cite{gee03}, a game in order to keep its user amused needs to:
\begin{enumerate*}[label=\roman*)]
 \item have a learning curve not too steep
 \item continuously give informations and hints
 \item remain continuously challeging during the gameplay without being too much
difficult
 \item engage the player as ``producer'' and not only a ``consumer''.
\end{enumerate*}
In this context, board games can result sometimes in long and mechanical 
operations that could ruin the overall gameplay. Pervasive games partially 
solves this problem providing digital instruments to speed up repetitive 
tasks and improving the game experience. This creates a new, immersive, 
gameplay opportunity preventing further burden to the player.
% TODO: add missing reference
As in [...] different attemps of merging board games with video games already 
exists, and its proved that there are advantages.
With these caracteristics, GameBank enables board games to become pervasive 
games, adding the virtual component needed~\cite{arango17}. Pervasive games are 
designed to blend real and virtual experience in only one.