\section{Introduction}
% no \IEEEPARstart

Playing games can be a social activity, especially when playing board ones. 
Sometimes, though, the fun can be reduced by the burden of managing the 
game itself by a master, that has to take in consideration the actions of 
everyone and act accordingly to the game mechanics. Recently a new concept of 
gaming, called pervasive gaming, is emerging. This concept merges the 
opportunities of the digital world and the physical world~\cite{arango17}. The 
most famous example of this is the mobile game Pokemon Go, published by 
Nintendo. Nintendo is continuing to develop this concept with the Nintendo Labo 
project: a physical extension for Nintendo Switch.
The virtualization improves the gameplay, making it easier to manage, even if 
there is always the need of a master~\cite{bjork01}.
For some activities, this role can be distributed to the members, spreading its 
duty to everyone. An activity that can be easily distribued, for example, is the 
one where there are transaction of in-game money. The distribution of this task 
can be performed in the same way Blockchain works: every memeber can emit a 
transaction and all the other peers have the possibility to check it and to 
determine its validity~\cite{nakamoto08}. This open up to a whole new type of 
possibilities, where new technologies, implemented as pervasive games, could be 
tested to real use cases.

Most of the time, players are located in the same room when playing board 
games. This particular scenario makes an already established technology in 
the market, Bluetooth, an optimal solution to implement pervasive applications. 
In our report, we will talk about GameBank, an example of pervasive game that, 
thanks to this wireless link, allows player to distribute the role of a Bank in 
monopoly-like games in a piconet.
Bluetooth, a technology developed initially by Ericsson, Nokia, IBM, Toshiba, 
Intel and many others~\cite{haartsen00}, has become today a widely used
standard especially in PAN (Personal Area Network) communications. This
wireless link has been chosen in GameBank for different reasons. First of all,
this protocol, being around for many years is known to most of the people,
making our application less naive for the avarage user. Secondly, Bluetooth can
cohexist with other protocols, and can handle interferences coming from other
signals (i.e. from WiFi). Finally, establishing the connection via Bluetooth
allows other wireless link to be used for other purposes, thus enabling the
user to perform other actions while playing our game, e.g. browsing the
internet via WiFi. This operation would not have been possible if we used 
WiFi-Direct, for example.\\

\todo{Set the correct references} This paper is organized as follows: Section 
CHANGEME describes how pervasive games and distribution of mastering duties can 
improve overall gameplay, while Section CHANGEME provides related 
works. Section CHANGEME describes how the application is designed and 
implemented, while Section CHANGEME lists the problems we faced while developing 
our application. At the end, in Section CHANGEME we analize some metrics results 
about Bluetooth and in Section CHANGEME we propose some future developments for 
our application. Finally, Section CHANGEME provides conclusions about our work.
